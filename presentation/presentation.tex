\documentclass[aspectratio=169,xcolor=dvipsnames]{beamer}

\usepackage[utf8]{inputenc}
\usepackage[ngerman]{babel}
\usepackage[T1]{fontenc}
\usepackage{amsmath}
\usepackage{amsfonts}
\usepackage{amssymb}
\usepackage{graphicx}
\usepackage{amsthm}
\usepackage{courier}
%\usepackage[left=0.1cm,right=0.1cm,top=1cm,bottom=1cm]{geometry}
\usepackage{xcolor}
\usepackage{enumerate}
\usepackage{todonotes}
\usepackage[babel,german=quotes]{csquotes}
\usepackage[style=numeric, backend=bibtex]{biblatex}
\usepackage{tikz}
\usepackage{hyperref}
\usepackage{listings}
\usepackage{verbatim}
\usepackage{xpatch}
\usepackage{lscape}
%\usepackage{mathptmx}
\usepackage{longtable}
\usepackage{pdfpages}
\usepackage{hyperref}
\usepackage{pifont}
%\usepackage[onehalfspacing]{setspace}
\usepackage{caption}
\captionsetup{font={scriptsize}}
\usepackage{multimedia}
\usepackage{media9}
\usepackage{mathrsfs}
\usepackage{bm}
\usepackage{etoolbox}

\usetheme{EastLansing}
%\usecolortheme{crane}
\usefonttheme{professionalfonts}
\useinnertheme{rounded}
\useoutertheme{default}

\newtheorem{satz}{Satz}
\newtheorem{law}{Gesetz}
\newtheorem{postulat}{Postulate}
\newtheorem{ex}{Beispiel}

\def\colorize<#1>{%
	\temporal<#1>{\color{red!50}}{\color{black}}{\color{black!50}}}
\setbeamertemplate{footline}[frame number, title, author]
\setbeamerfont{caption}{size=\scriptsize}
\setbeamerfont{footnote}{size=\scriptsize}
\setbeamercolor{item}{use={structure,normal text},fg=structure.fg!0!normal text.fg}
\setbeamercolor{special text}{use={1,0,0}}
\setbeamertemplate{navigation symbols}{\insertframenavigationsymbol\insertbackfindforwardnavigationsymbol}
\setbeamercolor{title}{fg=black!70!white,bg=green!10!yellow}
\setbeamercolor*{palette primary}{fg=green!10!yellow,bg=black!30!white}
\setbeamercolor*{palette secondary}{use=structure,fg=green!10!yellow,bg=black!50!white}
\setbeamercolor*{palette tertiary}{use=structure,fg=green!10!yellow,bg=black!70!white}
\setbeamercolor{frametitle}{fg=black!70!white,bg=green!10!yellow!70!white}
\setbeamertemplate{headline}
{%
	\begin{beamercolorbox}{section in head/foot}
		\vskip2pt\insertnavigation{\paperwidth}\vskip2pt
	\end{beamercolorbox}%
}
\setbeamercolor{block title}{fg=black!70!white,bg=green!10!yellow}
\setbeamercolor{block body}{use=structure,fg=black,bg=green!10!yellow!30!white}
\setbeamercolor{codex}{fg=black,bg=green!10!yellow!30!white}

%the following lines generate the bullet points in the header
\usepackage{remreset}
\makeatletter
%\@removefromreset{subsection}{section}
%\makeatother
%\setcounter{subsection}{1}


\title[Bayes-Statistik \& Wahlprognosen]{\textbf{Tag 5: Bayes-Statistik und Wahlprognosen}}
\author[Philip \and Tanja \and Maximilian]{Philip Biegel \and Tanja Bien \and Maximilian Wiesmann}
\institute[]{Sommerakademie der Studienstiftung Olang - Arbeitsgruppe 4: Empirische Wahlforschung und Wahlprognosen}
\date{6. September 2019}


\begin{document}
	
\begin{frame}
\titlepage
\end{frame}

\begin{frame}
\frametitle{Inhalt}
\section[Inhalt]{Inhalt}
\tableofcontents
\end{frame}

\section{Bayessche Inferenz I}

\begin{frame}
\frametitle{Bayessche Inferenz}
\begin{itemize}
\item[]<1-> 
\begin{beamercolorbox}[sep=0.5em,wd=\textwidth,shadow=true,rounded=true]{codex}
	\textit{\glqq At the heart of statistics lie the ideas of statistical inference. Methods of statistical inference enable the investigator to argue from the particular observations in a sample to the general case. In contrast to logical deductions made from the general case to the specific case, a statistical inference can sometimes be incorrect. Nevertheless, one of the great intellectual advances of the twentieth century is the realization that strong scientific evidence can be developed on the basis of many, highly variable, observations.\grqq} (aus: Encyclopedia of Mathematics)
\end{beamercolorbox}
\item[]<2->
\begin{satz}[Bayes]
	$$Pr(A|B) = \frac{Pr(B|A)Pr(A)}{Pr(B)} = \frac{Pr(B|A)Pr(A)}{\sum Pr(B|A)Pr(A)}$$
\end{satz}
\end{itemize}
\end{frame}

\begin{frame}
\frametitle{Bayesscher Wasserfall}
\begin{figure}
	\includegraphics[height=0.75\textheight]{Bayes}
	\caption{arbital.com}
\end{figure}
\end{frame}

\begin{frame}
\frametitle{Bsp.: Diagnosetest}
\begin{itemize}
\item<1-> $D+$: Person ist krank; $D-$: Person nicht krank
\item<1-> $T+$: Test positiv; $T-$: Test negativ
\item<1-> $Pr(T+|D+)=Pr(T-|D-)=90\%$
\item<1-> $Pr(D+)=1\%$
\item[]<2-> \begin{align*}
\Rightarrow Pr(T+) & = Pr(T+|D+)Pr(D+) + Pr(T+|D-)Pr(D-)\\ & = Pr(T+|D+)Pr(D+) + (1-Pr(T+|D+))(1-Pr(D-))\\ & = 0,108
\end{align*}
\item[]<3-> $$\overset{\text{Bayes}}{\Rightarrow} Pr(D+|T+) = \frac{Pr(T+|D+)Pr(D+)}{Pr(T+)}\approx 0,083$$
\end{itemize}
\end{frame}

\begin{frame}
\frametitle{Bayes für kontinuierliche Verteilungen}
\begin{itemize}
\item[]<1-> \begin{satz}[Bayes]
$$f(\theta|x)=\frac{f(x|\theta)f(\theta)}{f(x)}=\frac{f(x|\theta)f(\theta)}{\int f(x|\theta)f(\theta)\text{d}\theta}\propto f(x|\theta)f(\theta)$$
\end{satz}
\item<2-> $f(\theta|x)$: posterior distribution
\item<2-> $f(\theta)$: prior distribution
\item<2-> $f(x|\theta)$: likelihood function (auch bez. mit $L(\theta)$)
\item<2-> $f(x)$: marginal likelihood
\end{itemize}
\end{frame}

\begin{frame}
\frametitle{Bayesian Point Estimates}
\begin{itemize}
\item<1-> posterior mean $E(\theta|x)=\int \theta f(\theta|x)\,\text{d}\theta$
\item<2-> posterior mode $Mod(\theta|x) = \underset{\theta}{\text{arg max}}\,f(\theta|x)$
\item<3-> posterior median $Med(\theta|x) = a, \text{sd. } \int_{-\infty}^{a}f(\theta|x)\,\text{d}\theta = \frac{1}{2} = \int_{a}^{\infty}f(\theta|x)\,\text{d}\theta$
\end{itemize}
\end{frame}

\begin{frame}
\frametitle{Credible Intervals}
\begin{itemize}
\item<1-> Das Intervall $I=[t_l,t_u]$ heißt $\gamma\cdot 100\%$ credible interval (Glaubwürdigkeitsintervall), falls gilt: $$\int_{t_l}^{t_u}f(\theta|x)\,\text{d}\theta = \gamma$$
\item<2-> equi-tailed: $$\int_{-\infty}^{t_l}f(\theta|x)\,\text{d}\theta = \int_{t_u}^{\infty}f(\theta|x)\,\text{d}\theta = \frac{1-\gamma}{2}$$
\item<3-> highest posterior density (HPD): $$f(\theta|x)\geq f(\tilde{\theta}|x)~\forall\theta\in I, \tilde{\theta}\notin I$$
in \glqq schönen\grqq~Fällen gilt: $f(t_l|x)=f(t_u|x)$
\end{itemize}
\end{frame}

\begin{frame}
\frametitle{Beta-Verteilung}
\begin{itemize}
\item<1-> $\theta \sim Be(\alpha,\beta)$
\item<2-> Dichtefunktion $$f(\theta)=\left\{\begin{array}{ll}
\frac{1}{B(\alpha,\beta)}\theta^{\alpha -1}(1-\theta)^{\beta -1} & \text{für }\theta\in[0,1]\\
0 & \text{sonst}\\
\end{array}\right.$$
mit Beta-Funktion $B(\alpha,\beta)=\int_0^1t^{\alpha-1}(1-t)^{\beta-1}\,\text{d}t = \frac{\Gamma(\alpha)\Gamma(\beta)}{\Gamma(\alpha+\beta)}$
\item<3-> $$E(\theta) = \frac{\alpha}{\alpha+\beta}\quad,\quad Mod(\theta)=\frac{\alpha-1}{\alpha+\beta-2}\quad,\quad Var(\theta)=\frac{\alpha\beta}{(\alpha+\beta)^2(\alpha+\beta+1)}$$
\item<4-> für $\alpha=\beta=1$ erhält man die Gleichverteilung auf $[0,1]$
\end{itemize}
\end{frame}

\begin{frame}
\frametitle{Beta-Verteilung}
\begin{figure}
\includegraphics[height=0.85\textheight]{betaDistribution}
\end{figure}
\end{frame}

\begin{frame}
\frametitle{Beta-Verteilung}
\begin{itemize}
\item<1-> $X\sim Bin(n,\theta)~,\quad \theta\sim Be(\alpha,\beta)$
\item<2-> $\Rightarrow f(\theta|x)\propto f(x|\theta)f(\theta)\propto \theta^x(1-\theta)^{n-x}\theta^{\alpha-1}(1-\theta)^{\beta-1}$
\item<3-> $\Rightarrow \theta|x \sim Be(\alpha+x,\beta+n-x)$
\item<4-> Beta-Verteilung \textit{konjugiert} zur Binomialverteilung
\end{itemize}
\end{frame}

\begin{frame}
\frametitle{Beta-Verteilung Beispiel}
\begin{columns}
\uncover<1,2,3,4>{\begin{column}{0.5\textwidth}
\begin{itemize}
\item<1-> $\theta\sim Be(3,2)~,\quad X\sim Bin(n,\theta)$
\item<2-> Ergebnis der Stichprobe mit Umfang $n=10$: $x=8$
\item<3-> $\Rightarrow \theta|x\sim Be(11,4)$
\end{itemize}
\end{column}}
\uncover<4>{\begin{column}{0.5\textwidth}
\begin{figure}
\begin{figure}\includegraphics[width=0.75\textwidth]{beta}\caption{Held, Bov\'{e}: Applied Statistical Inference, S. 173}\end{figure}
\end{figure}
\end{column}}
\end{columns}
\end{frame}

\begin{frame}
\frametitle{Capture-Recapture-Method}
\begin{itemize}
\item<1-> Möchte Anzahl $N$ von Fischen in einem See bestimmen
\item<2-> Entnehme $M$ Fische, markiere diese und gebe sie wieder zurück
\item<3-> Entnehme $n$ Fische und bestimme Anzahl $x$ der markierten Fische
\item<4-> Möglicher Ansatz: $\frac{M}{N}\approx \frac{x}{n}$
\end{itemize}
\end{frame}

\begin{frame}
\frametitle{Capture-Recapture-Method}
\begin{itemize}
\item<1-> Ansatz mittels Bayesscher Inferenz:
\item<2-> Likelihood: $$f(x|N)=\frac{\binom{M}{x}\binom{N-M}{n-x}}{\binom{N}{n}}$$
\item<3-> Geometrische Verteilung als Prior: $f(N)\propto \pi(1-\pi)^{N-1}$
\item<4-> Bsp.: $M=26\, ,~n=63\, ,~x=5\, ,~\pi=0,0011$
\end{itemize}
\end{frame}

\begin{frame}
\frametitle{Capture-Recapture-Method}
\begin{columns}
\begin{column}{0.5\textwidth}
\begin{figure}
\includegraphics[width=0.9\textwidth]{geom}
\caption{Held, Bov\'{e}: Applied Statistical Inference, S. 179}
\end{figure}
\end{column}
\begin{column}{0.5\textwidth}
\begin{figure}
\includegraphics[width=0.9\textwidth]{capture-recapture}
\caption{ebd.}
\end{figure}
\end{column}
\end{columns}
\begin{itemize}
\item<1-> $Mod(N|x)=313\, ,~Med(N|x)=392\, ,~E(N|x)=446.5$
\item<2-> $95\%$ HPD Intervall: $[61; 869]$
\item<3-> Erster Ansatz liefert: $N\approx 327,6$
\end{itemize}
\end{frame}

\begin{frame}
\frametitle{Variablentransformation}
\begin{itemize}
\item $Y=g(X)$
\item $$f_Y(y)=f_X(g^{-1}(y))\left|\frac{\text{d}g^{-1}(y)}{\text{d}y}\right|$$
\end{itemize}
\end{frame}

\begin{frame}
\frametitle{Jeffrey's prior}
\begin{itemize}
\item<1-> Ziel: Priori Verteilung invariant unter Variablentransformation
\item<2-> Fisher Information: $$I(\theta)=-\frac{\text{d}^2l(\theta)}{\text{d}\theta^2}$$
mit log-likelihood function $l(\theta)=ln(L(\theta))$
\item<3-> Erwartete Fisher Information: $$J(\theta)=E(I(\theta;X))$$
wobei der Erwartungswert bezüglich $X$ gebildet wird
\item<4-> Jeffrey's prior $$f(\theta)\propto\sqrt{J(\theta)}$$
\item<5-> drückt Uninformiertheit aus
\end{itemize}
\end{frame}

\begin{frame}
\frametitle{Jeffrey's prior}
\uncover<1,2,3>{
\begin{itemize}
\item<1-> Für multivariate Verteilungen: Erwartete Fisher Information Matrix $$[J(\boldsymbol{\theta})]_{ij}=E(-\partial_i\partial_jl(\boldsymbol{\theta}))$$
\item<2-> Jeffrey's prior $$f(\boldsymbol{\theta})\propto\sqrt{\text{det}\,J(\boldsymbol{\theta})}$$
\end{itemize}}
\uncover<3>{\begin{satz}
Jeffrey's prior ist invariant unter Variablentransformation, d.h. wenn $\phi = g(\theta)$ und $f_{\theta}(\theta)\propto \sqrt{J_{\theta}(\theta)}$, dann ist $f_{\phi}(\phi)\propto \sqrt{J_{\phi}(\phi)}$.
\end{satz}}
\end{frame}

\begin{frame}
\frametitle{Jeffrey's prior für $Bin(n,\theta)$}
\begin{itemize}
\item<1-> $X\sim Bin(n,\theta)$, also $f(x|\theta)= \binom{n}{x}\theta^x(1-\theta)^{n-x}$
\item<2-> $l(\theta)=x\,ln(\theta)+(n-x)\,ln(1-\theta)+const$
\item<3-> $-\frac{\text{d}^2l(\theta)}{\text{d}\theta^2}=\frac{x}{\theta^2}-\frac{n-x}{(1-\theta)^2}$ 
\item<4-> $E(-\frac{\text{d}^2l(\theta)}{\text{d}\theta^2}) = \frac{n\theta}{\theta^2}-\frac{n(1-\theta)}{(1-\theta)^2}=\frac{n}{\theta(1-\theta)}$
\item<5-> $\Rightarrow \sqrt{J(\theta)}\propto Be(\frac{1}{2},\frac{1}{2})$
\end{itemize}
\end{frame}

\begin{frame}
\frametitle{Jeffrey's prior für Multinomialverteilung}
\begin{itemize}
\item<1-> $\boldsymbol{x}=(x_1,\dots,x_p)\,,~\boldsymbol{\theta}=(\theta_1,\dots,\theta_p)$ mit $\sum_{i=1}^p\theta_p = 1$
\item<2-> Multinomialverteilung $M_p(n,\boldsymbol{\theta})$: $$f(\boldsymbol{x})=\left\{\begin{array}{ll}
\frac{n!}{\prod_{i=1}^{p}x_i!}\prod_{i=1}^{p}\theta_i^{x_i} & \text{falls }\sum_{i=1}^{p}x_i=n\\
0 & \text{sonst}
\end{array}\right.$$
\item<3-> Dirichlet-Verteilung $D(\alpha_1,\dots,\alpha_p)$: $$f(\boldsymbol{\theta})\propto\prod_{i=1}^{p}\theta_i^{\alpha_i-1}$$
\item<4-> Jeffrey's prior für $M_p(n,\boldsymbol{\theta})$:
$$\boldsymbol{\theta}\sim D\left(\frac{1}{2},\dots,\frac{1}{2}\right)$$
\end{itemize}
\end{frame}

\section{KOALA}

\begin{frame}
\frametitle{KOALA}
\begin{itemize}
\item Ansatz für Mehr-Parteien-Systeme mit besonderem Fokus auf Koalitionswahrscheinlichkeiten
\item Meinungsbild (wenn die Wahl heute wäre) 
\item Monte-Carlo-Ansatz basierend auf einem Bayesschen Multinomial-Dirichlet Model
\item Vorstellung neuer Visualisierungstechnik
\end{itemize}
\end{frame}

\begin{frame}
\frametitle{KOALA}
\begin{columns}
\begin{column}{0.5\textwidth}
\begin{figure}[t!]
\includegraphics[height=0.5\textheight]{partyshare}
\caption{Stimmanteile der Parteien}
\end{figure}
\end{column}
\begin{column}{0.5\textwidth}
\begin{figure}[h!]
\includegraphics[height=0.5\textheight]{poe}
\caption{Sitzanteil für Koalition}
\end{figure}
\end{column}
\end{columns}
\end{frame}

\begin{frame}
\frametitle{KOALA}
\begin{columns}
\begin{column}{0.6\textwidth}
Klassische Wahlberichterstattung
\begin{itemize}
\item Darstellung der Stimmanteile der Parteien
\item Fokus auf Ergebnisse einzelner Parteien
\item Unzureichende Vermittlung von Unsicherheit			
\end{itemize}
\end{column}
\begin{column}{0.4\textwidth}
\begin{figure}
\includegraphics[width=\textwidth]{partyshare}
\end{figure}
\end{column}
\end{columns}
\end{frame}

\begin{frame}
\frametitle{KOALA}
\begin{columns}
\begin{column}{0.6\textwidth}
Vorgeschlagene Wahlberichtserstattung
\begin{itemize}
\item Darstellung der Wahrscheinlichkeit eines Ereignisses durch Dichteplots
\item Fokus auf spezifischem Ereignis (z.\,B. Koalitionsbildung)
\item Vermittlung der Unsicherheit und Bandbreite der Ergebnisse
\end{itemize}
\end{column}
\begin{column}{0.4\textwidth}
\begin{figure}
\includegraphics[width=\textwidth]{poe}
\end{figure}
\end{column}
\end{columns}
\end{frame}

\begin{frame}
\frametitle{KOALA}
\begin{columns}
\begin{column}{0.4\textwidth}
Typische Überschrift:
\begin{itemize}
\item \glqq Union-FDP verliert die Mehrheit\grqq{}
\end{itemize}

\vspace{1cm}
Vorgeschlagene Überschrift:
\begin{itemize}
\item \glqq Union-FDP erreicht zu 26\% Sitzmehrheit \grqq{}
\item \glqq FDP zieht zu 51\% ins Parlament ein \grqq{}
\end{itemize}

\end{column}
\begin{column}{0.6\textwidth}
\begin{figure}
\includegraphics[height=0.35\textheight]{partyshare}
\end{figure}
\begin{figure}
\includegraphics[height=0.40\textheight]{poe}
\end{figure}
\end{column}
\end{columns}
\end{frame}

\section{Bayessche Inferenz II}

\begin{frame}
\frametitle{Asymptotisches Verhalten}
\uncover<1,2>{
\begin{itemize}
\item<1-> Parametermenge $\Theta=\{\theta_0,\theta_1,\dots\}$, $\theta_0$ ist wahrer Parameter, $X_{1:n}$ Stichprobe von Verteilung mit Dichtefunktion $f(x|\theta)\,,\,\theta\in\Theta$. Dann gilt:
$$\underset{n\rightarrow\infty}{\text{lim}}f(\theta_0|x_{1:n})=1\quad \text{und}\quad \underset{n\rightarrow\infty}{\text{lim}}f(\theta_i|x_{1:n})=0~\forall i\neq 0$$
\end{itemize}}
\uncover<2>{\begin{figure}
\includegraphics[width=0.3\textwidth]{beta2}
\end{figure}}
\end{frame}

\begin{frame}
\frametitle{Empirische Bayes Methoden}
\begin{itemize}
\item<1-> Wählen Priori Verteilung in Abhängigkeit von erhaltenen Daten
\item<2-> Bsp.: Analyse der Trefferquoten von Baseballspielern:
\item<3-> Spieler A hat 4 von 10 Bällen getroffen, Spieler B hat 300 von 1000 Bällen getroffen. Welcher Spieler ist besser?
\item<4-> Trefferzahl $X\sim Bin(n,\theta)$, Treffwahrscheinlichkeit $\theta\sim Be(\alpha,\beta)$
\end{itemize}
\end{frame}

\begin{frame}
\frametitle{Empirische Bayes Methoden}
\begin{columns}
\uncover<1,2,3,4,5,6>{\begin{column}{0.6\textwidth}
\begin{itemize}
\item<1-> Bilde Priori Verteilung durch Interpolation über Werte aller Baseballspieler
\item<3-> $\Rightarrow \theta\sim Be(78,661\,;\,224,875)$
\item<4-> Posteriori Trefferverteilung für Spieler A: $\theta|x\sim Be(82,661\,;\,230,875)$ hat Erwartungswert 0,264
\item<5-> Posteriori Trefferverteilung für Spieler B: $\theta|x\sim Be(378,661\,;\,924,875)$ hat Erwartungswert 0,29
\item<6-> Shrinkage
\end{itemize}
\end{column}}
\uncover<2,3,4,5,6>{\begin{column}{0.4\textwidth}
\begin{figure}\includegraphics[width=\textwidth]{baseball}\caption{varianceexplained.org}\end{figure}
\end{column}}
\end{columns}
\end{frame}

\section{Vorhersagemodell}

\begin{frame}
	\frametitle{Vorhersagemodell}
		\begin{itemize}
			\item Dynamisches Bayessches Vorhersagemodell für Mehr-Parteien-Wahlen 
			\item Kombination aus Regressionsmodell und Bayesschen Modell
			\item Vorhersage von Stimmanteilen und weiterer Kennzahlen
		\end{itemize}
\end{frame}

\begin{frame}
	\frametitle{Regressionsmodell}
	\begin{itemize}
		\item Vorhersage der Stimmenanteile lange vor der Wahl
		\item Nutzung von Daten vorheriger Wahlen
		\item Vorhersager $X$ sagt Ergebnisse $V$ mit Parametern $\theta$ vorher
	\end{itemize}
\end{frame}

\begin{frame}
	\frametitle{Bayessches Modell}
	\begin{itemize}
		\item<1 -> Vorhersage der aktuellen Stimmenanteile
		\item<1 -> Nutzung von Daten aus aktuellen Umfragen
		\item<2 -> Backward Random Walk: $\alpha_t = \alpha_{t+1} + \omega$
		\item<3 -> Stimmanteile am Wahltag = Posterior predictive distribution des Regressionsmodells
		\item<4 -> Update bei neuen Ereignissen
	\end{itemize}
\end{frame}

\begin{frame}
	\frametitle{Vorhersagemodell}
	\begin{figure}
		\includegraphics[height=0.85\textheight]{development}
	\end{figure}
\end{frame}

\begin{frame}
	\frametitle{Vorhersagemodell}
	\begin{figure}
		\includegraphics[height=0.85\textheight]{forecast}
	\end{figure}
\end{frame}

\section{Bibliographie}
\begin{frame}
\frametitle{Bibliographie}
\begin{itemize}
\item Bauer, Bender, Klima, Küchenhoff: KOALA: a new paradigm
for election coverage. An opinion poll-based now-cast of probabilities of events in multi-party electoral systems.
\item Held, Bov\'{e}: Applied Statistical Inference
\item Stoetzer, Neunhoeffer, Gschwend, Munzert, Sternberg: Forecasting elections in multiparty systems: A bayesian approach combining polls and fundamentals.
\end{itemize}
\end{frame}

\end{document}
